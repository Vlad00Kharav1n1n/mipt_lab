\documentclass[a4paper, 12pt]{article}
\usepackage{cmap}
\usepackage[warn]{mathtext}
\usepackage[T2A] {fontenc}
\usepackage[utf8]{inputenc}
\usepackage[english,russian]{babel}
\usepackage{graphicx}
\graphicspath{{./images/}}
\author{Владислав Харавинин Б02-113}
\title{Лабораторная работа №1.4.2 "Определение ускорения свободного падения при помощи  оборотного маятника" }
\date{\today}
\begin{document}	
	\maketitle


	\textbf{Цель работы:}\\
С помощью оборотного маятника измерить величину ускорения свободного падения.\\

\textbf{В работе используются:}\\
Оборотный маятник с двумя подвесными призмами и двумя грузами(чечевицами); электронный счетчик времени и числа колебаний; подставка с острием для определения центра масс маятника; электронные весы; закрепленная на стене консоль для подвешивания маятника; металлические линейки; штангенциркуль длиной 1 м.\\\\
\textbf{Теоретическая часть}\\
\textit{Физическим маятником} называют твердое тело, способное совершать колебания в вертикальной плоскости, будучи подвешено за одну из своих точек в поле тяжести. Ось, проходящая через точку подвеса перпендикулярно плоскости качания, называется \textit{осью качания} маятника. \\
\begin{figure}[h]
	\centering
	\includegraphics[scale=0.3]{stick}
	\caption{}
\end{figure}\\
Стержень подвешен на расстоянии а от центра масс. Следовательно \textit{плечо} силы тяжести относительно вертикальной оси равно $a*\sin\phi$. При \textit{малых} углах момент силы тяжести $M = -mg*\sin\phi = -mga\phi $, при \textit{малых} амплитудах отклонения движение будет иметь характер \textit{гармонических колебаний}.\\
По формуле (2) и пониманию того, что $\frac{d\omega}{dt} = \ddot{\phi} $ находим, что
\begin{equation}
	J\ddot{\phi} + mga\phi = 0
\end{equation}
Решением этого уравнения является $\phi(t)=A*\cos(\Omega*t+\alpha)$, где $\Omega=\sqrt{\frac{mga}{J}}$. Зная что $T*\Omega=2\pi$, получим 
\begin{equation}
	T=2\pi\sqrt{\frac{J}{mga}}
\end{equation}\\
Сравнивая полученную формулу с известной формулой колебания математического маятника  $(T = 2\pi\sqrt{\frac{l}{g}})$ можно определить \textit{приведенную длину} физического маятника как

\begin{equation}
l_{пр} = \frac{J}{ml}
\end{equation}\\
\textbf{Теорема Гюйгенса об оборотном маятнике}\\
\begin{figure}[h!]
	\centering
	\includegraphics[scale=0.5]{Huihense}
	\caption{}
\end{figure}\\
Пусть $O_1$ точка подвеса физического маятника, 
C - его центр масс. Отложим отрезок длиной $l_{пр}$ вдоль линии $O_1$C, и обозначим соответсвующую точку как $O_2$ - эту точку называют \textit{центром качания} физического маятника. Заметим, что приведенная длина всегда больше больше расстояния до центра масс ($l_{пр} > l$), поэтому точка $O_2$ лежит по другую сторону от центра масс.\\ Точки $O_1$ и $O_2$ обладают свойством взаимности: если перевернуть маятник и подвесить его за точку $O_2$, то его период малых колебаний останется таким же, как и при подвешивании за точку $O_1$ (теоема Гюйгенса).\\
Докажем \textit{теорему Гюйгенса} об оборотном маятнике. Пусть $O_1$ и $O_2$ - две точки подвеса физического маятника, лежащие на одной прямой с точкой C по разные стороны от нее. Тогда периоды колебаний маятника равны соответсвенно
\begin{equation}
	T_1 = 2\pi\sqrt{\frac{J_1}{mgl_1}} \hspace{16pt}
	T_2 = 2\pi\sqrt{\frac{J_2}{mgl_2}}
\end{equation} \\
По теореме Гюйгенса-Штейнера имеем
\begin{equation}
	J_1 = J_C + ml_1^2 \hspace{16pt} J_2 = J_C + ml_2^2	
\end{equation} \\
где $J_C$ - момент инерции маятника относительно оси, проходящей через центр масс перпендикулярно плоскости качания.\\
Пусть периоды колебаний одинаковы: $T_1 = T_2$. Тогда одинаковы должны быть и приведенные длины:
$$ l_{пр} = \frac{J_1}{ml_1} = \frac{J_2}{ml_2} = \frac{J_C}{ml_1} + l_1 = \frac{J_C}{ml_2} + l_2$$
Отсюда следует, что при $l_1 \neq l_2$ справедливо равенство 
\begin{equation}
	J_C = ml_1l_2
\end{equation}
Откуда получим 
\begin{equation}
	l_{пр} = l_1 + l_2
\end{equation}
Таким образом если периоды колебаний при подвешивании маятника в точках $O_1$ и $O_2$ равны, то расстояние между точками подвеса равно приведенной длине маятника. Обратное тоже верно. \\\\\\








\textbf{Предварительный расчет положения грузов}
\begin{figure}[h!]
	\centering
	\includegraphics[scale=0.7]{spinner}
	\caption{}
\end{figure}\\
Расчет с использованием моментов инерции относительно подвеса.\\ В данном методе все моменты инерции вычисляются относительно точки подвеса $П_2$. Положение центра масс ($l_1$ и $l_2$) считается заданным, вычисляются соответствующие положения грузов $b_1$ и $b_2$.\\Используются следующие соотношения:\\
\begin{equation}
Ml_1 = m_{ст}\frac{L}{2} + m_{пр2}L + m_1b_1 + m_2(b_2 + L)
\end{equation}
момент инерции тонкого стержня длиной $l_{ст}$ с призмами:
$$ J_{ст} = m_{ст}(\frac{l_{ст}^2}{12} +\frac{L^2}{4}) + m_{пр2}L^2$$
момент инерции грузов на стрежне:
$$J_{гр} = m_1(L - b_1)^2 + m_2b_2^2 $$
суммарный момент инерции всего маятника определяется из формулы:
$$J_П = MLl_2 = J_{ст} + J_{гр}.$$
Расчет проводится по следующему алгоритму:\\
1.	Выбираем и фиксируем некоторое значение расстояния $l_2$.\\
2.	По известным массам расчитываем $J_П$ и $J_{ст}$, которые зависят только от величин $l_2$ и $L$.\\
3.	Варьируя длину $b_2$ в пределах от $b_{2min} = 0$ до
$b_{2max} = (l -L)/2$, по формуле (8) вычислим сооответсвующее положение первого груза $b_1$, а затем момент инерции грузов $J_{гр}$.\\
4.	Строим график $J_{гр}$($b_2$) и по точке пересечения с уровнем $J_{гр}$ = $J_{П}$ - $J_{ст}$ определяем положения грузов $b_2$ и $b_1$.\\\\
\textbf{Оценка погрешностей}\\
Оценим влияние погрешностей измерений на точность расчетов по формуле:
\begin{equation}
 g = (2\pi)^2\frac{l_1^2 - l_2^2}{T_1^2l_1 - T_2^2l_2}	
\end{equation}
что также можно представить как
\begin{equation}
	g = g_0\cdot\frac{\lambda - 1}{\lambda-\frac{T_2^2}{T_1^2}}	\hspace{10pt} где\hspace{10pt}  g_0 = (2\pi)^2\frac{L}{T^2} \hspace{10pt} и \hspace{10pt} \lambda = \frac{l_1}{l_2}
\end{equation}
Используя малость величины $\varepsilon = \frac{\Delta T}{T}\ll 1$ и отличие от еденицы величины лямбда $\lambda\neq 1$ получим:
\begin{equation}
	g = g_0\cdot\frac{\lambda - 1}{\lambda-(1 + \varepsilon)^2}\approx g_0\cdot\frac{1}{1- \frac{2\varepsilon}{\lambda - 1}}\approx g_0\cdot(1 + 2\beta\varepsilon) \hspace{10pt} где\hspace{10pt} \beta= \frac{1}{\lambda -1}=\frac{l_2}{l_1-l_2} 
\end{equation}
Пусть все периоды измерены с одинаковой погрешностью $\sigma_T$, расстояние L между точками подвеса с погрешностью $\sigma_l$. Погрешность определения величины $g_0$ по формуле математического маятника
\begin{equation}
\frac{\sigma_{g_0}}{g_0} = \sqrt{(\frac{\sigma_l}{L})^2 + 4(\frac{\sigma_T}{T})^2}
\end{equation}
Это - основная погрешность опыта. Видно, что для ее минимизации необходимо максимально точно измерить расстояние между точками подвеса L и период колебания T.\\
Используемая нами в этом опыте погрешность измерения длины $\sigma_l\approx 1мм$, тогда при $L\approx1 м$ точность составляет примерно $0.1\%$. Погрешность периода может быть уменьшена за счет увеличения времени измерения. При измерении электронным счетчиком погрешность измерения полного времени составляет порядка $\sigma_t = 0.01c$. Тогда общая погрешность g не превысит $0.1\%$, если погрешность измерения времени будет на порядок меньше, чем погрешность измерения длины. Тогда проведем n = 100 колебаний.\\
Проанализируем влияние прибавки $g = g_0 + \Delta g $, где согласно формуле(11)
$$ \Delta g\approx\frac{2l_2}{l_1 - l_2}\frac{\Delta T}{T}g_0$$
Достаточно учесть, что основной вклад в относительную погрешность $\Delta g$ вносят величины $\Delta T$ и $\Delta l = l_1 - l_2$, поскольку являются разностями двух близких величин. Поэтому
$$ \frac{\sigma_{\Delta g}}{g}\approx\frac{2\beta\Delta T}{T}\sqrt{(\frac{\sigma_{\Delta T}}{\Delta T})^2 + (\frac{\sigma_{\Delta l}}{\Delta l})^2} = \sqrt{8(\beta\frac{\sigma_T}{T})^2 + 8(\beta\frac{\Delta T}{T}\frac{\sigma_l}{\Delta l})^2}$$
Тогда для полной погрешности получим
\begin{equation}
\frac{\sigma_{g}}{g}\approx\sqrt{(\frac{\sigma_L}{L})^2 + 4(\frac{\sigma_T}{T})^2 + 8(\beta\frac{\sigma_T}{T})^2 + 8(\beta\frac{\Delta T}{T}\frac{\sigma_l}{\Delta l})^2}
\end{equation}
  



\textbf{Ход работы}\\
\textbf{1.} Измерение массы электрическими весами не имеет случайной погрешности. Систематическая погрешность весов $\sigma^{сист}_m = 0.1 г$.\\
\begin{table}[h!]
	\textbf{Таблица 1. Измерение массы стержня, призм и грузов}\\
	\centering
	\begin{center}
		\begin{tabular}{|c|c|c|c|c|}
			\hline
			м. стержня, $m_{ст}$ & м. призмы1, $m_{пр1}$ & м. призмы2, $m_{пр2}$& м. груза1, $m_{1}$& м. груза2, $m_{2}$\\
			\hline
			1019.4, г  & 60.5, г & 72.7, г & 1484.9, г & 1481.8, г\\
			\hline
		\end{tabular}
	\end{center}
\end{table}\\
\textbf{2.} Закрепим подвесные призмы симметрично на стержне, так чтобы ребра призм стали параллельны и "смотрят" в сторону центра маятника.
\textbf{3.} Измерение длины штангенциркулем не имеет случайной погрешности. Систематическая погрешность штангенциркуля $\sigma^{сист}_l = 0.1 см$.\\
Расстояние между призмами L = 55.3  $\pm 0.1 $ см\\
\textbf{4.}  Пусть $\lambda = l_1/l_2 = 2.31$. Откуда очевидно получим $l_1 = 38.6\hspace{10pt} см$, $l_2 = 16.7 \hspace{10pt}см$, $ b_1 = 20.1\hspace{10pt} см$, $b_2 = 10.1\hspace{10pt} см$. 
Закрепим призмы и чечевицы в соответствующих местах.\\
\textbf{5.}
С помощью Т-образной подставки определим положение центра масс с грузами. Убедимся в том, расчет по формуле(8) совпадает с реальными значениями.\\
\textbf{6.} Убедимся в работоспособности системы. Маятник при качании не касается элементов установки и не проскальзывает в подвесе. Счетчик корректно считает число колебаний и их время.\\
\textbf{7.} Проведем серию измерений периода маятника с одной стороны.\\
\begin{table}[h!]
	\textbf{Таблица 2. Измерение периода маятника с одной стороны}\\
	\centering
	\begin{center}
		\begin{tabular}{|c|c|c|c|}
			\hline
			№ опыта & 1 & 2 & 3 \\
			\hline
			Период, $Т_2$, с  & 1.495 & 1.500 & 1.495 \\
			\hline
		\end{tabular}
	\end{center}
\end{table}\\
\textbf{8.} Проведем серию измерений периода маятника с другой стороны.\\
\begin{table}[h]
	\textbf{Таблица 3. Измерение периода маятника с другой стороны}\\
	\centering
	\begin{center}
		\begin{tabular}{|c|c|c|c|}
			\hline
			№ опыта & 1 & 2 & 3 \\
			\hline
			Период, $Т_1$, с  & 1.495 & 1.495 & 1.495 \\
			\hline
		\end{tabular}
	\end{center}
\end{table}\\ 
\textbf{9.} 
Заметим, что значения периодов отличаются менее чем на $1\%$.\\
\textbf{10.}
Согласно нашей оценке мы провели измерение периодов $T_1$ и $T_2$ с n = 100 числом колебаний.\\
$$ t_1 = 149.39\pm 0.01c \hspace{10pt} t_1 = 149.50 \pm 0.01c \hspace{10pt} T_1 = \frac{t_1}{n} = 1.494 c \hspace{10pt} T_2 = \frac{t_2}{n} = 1.495 c$$
\textbf{11.}
Положение грузов не изменялось, следовательно измерения проведены корректно.\\
\textbf{12.}
$$ g_1 = (2\pi)^2\frac{L}{T^2} \hspace{5pt},где\hspace{5pt} T = 0.5(T_1 + T_2) \hspace{15pt} g_1 = (2\pi)^2\frac{0.553\hspace{5pt}м}{(1.4945\hspace{5pt} с)^2} = 9.78\pm 0.01 \hspace{10pt} \frac{м}{с^2} $$

$$  g_2 = (2\pi)^2\frac{l_1^2 - l_2^2}{T_1^2l_1 - T_2^2l_2}\hspace{15pt}g_2 = (2\pi)^2\frac{0.386^2 - 0.167^2}{1.494^2\cdot0.386 - 1.495^2\cdot0.167}\frac{м}{с^2} = 9.79\pm0.01\hspace{5pt} \frac{м}{с^2} $$

$$ g_3 = (2\pi)^2\frac{L}{T^2}(1 + \frac{2\cdot l_2}{l_1 - l_2}\frac{\Delta T}{T}),\hspace{5pt} где\hspace{5pt} T = T_1,\:а\hspace{5pt}\Delta T = (T_2 - T_1)\hspace{15pt}$$

$$ g_3 = (2\pi)^2\frac{0.553}{1.494^2}(1 + \frac{2\cdot0.167}{0.386 - 0.167}\frac{0.001}{1.494})\frac{м}{с^2} = 9.79 \pm 0.01 \hspace{5pt} \frac{м}{с^2}$$

Все результаты отличаются от табличного $g = 9,806 \frac{м}{с^2}$. Но лежат в диапазоне "трех сигм". Что говорит о существовании неучтенной систематической погрешности. Расчеты опирающиеся на более глубокую теорию (теорему Гюйгенса - Штейнера и теорему Гюйгенса об оборотном маятнике) и учитывающие невозможность добиться совпадения периодов показывают большую точность. \\
Трение не могло стать таким параметром, так как амплитуда уменьшилась вдвое только при числе колебаний N = 500. Добротность системы 
 $Q =\frac{\pi N}{\ln2} \approx 2300 $.\\\\\\
\textbf{Вывод}\\\\
С помощью оборотного маятника мы измерили величину ускорения свбодного падения. В качесте окончательного результата возьмем тот, что глубже опирается на теорию, а значит, обладает большей достоверностью.
 $$g = g_2 = 9.79 \pm 0.01 \hspace{10pt} \frac{м}{с^2} $$
Результат отличается от табличного, но лежит в диапазоне "двух сигм". Что говорит о существовании неучтенной систематической погрешности. Трение не могло стать таким параметром. Возможно нужно учесть невозможность абсолютно точного измерения (из-за трудности достижения параллельности) штангенциркулем и увеличить погрешность измерения получаемого с его использованием. Или пересмотреть методы расчета величины ускорения свободного падения.
 
 
\end{document}
